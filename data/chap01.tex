% !TeX root = ../thuthesis-example.tex

\chapter{引言}

%大模型背景


\section{论文的语言及表述}

1. 大模型能力观察
Won't get fooled again: Answering questions with false premises

2.1 高效方法之一 -- 科学化预测
Predicting Emergent Abilities with Infinite Resolution Evaluation

2.2 高效方法之二 -- 可扩展训练策略
Minicpm: Unveiling the potential of small language models with scalable training strategies
A Multi-Power Law for Loss Curve Prediction Across Learning Rate Schedules (部分)

2.3 高效方法之数据 -- 精选数据让模型
DecorateLM: Data Engineering through Corpus Rating, Tagging, and Editing with Language Models


3.1 大模型能力高效利用
Knowledgeable prompt-tuning: Incorporating knowledge into prompt verbalizer for text classification
Sparse structure search for delta tuning

4.1 产出的工具包
OpenDelta: A plug-and-play library for parameter-efficient adaptation of pre-trained models


\section{论文题目的写法}

论文题目应简明扼要地反映论文工作的主要内容,力求精炼、准确,切忌笼统。
论文题目是对研究对象的准确、具体描述,一般要在一定程度上体现研究结论,因此,论文题目不仅应告诉读者这本论文研究了什么问题,更要告诉读者这个研究得出的结论。
例如:“在事实与虚构之间:梅乐、卡彭特、沃尔夫的新闻观”就比“三个美国作家的新闻观研究”更专业、更准确。



\section{摘要的写法}

论文摘要是对论文研究内容的高度概括,应具有独立性和自含性,即应是 一篇简短但意义完整的文章。
通过阅读论文摘要,读者应该能够对论文的研究 方法及结论有一个整体性的了解,因此摘要的写法应力求精确简明。
论文摘要 应包括对问题及研究目的的描述、对使用的方法和研究过程进行的简要介绍、 对研究结论的高度凝练等,重点是结果和结论。

论文摘要切忌写成全文的提纲,尤其要避免“第 1 章……;第 2 章……;……”这样的陈述方式。



\section{引言的写法}

一篇学位论文的引言大致包含如下几个部分:
1、问题的提出;
2、选题背 景及意义;
3、文献综述;
4、研究方法;
5、论文结构安排。
\begin{itemize}
  \item 问题的提出:要清晰地阐述所要研究的问题“是什么”。
    \footnote{选题时切记要有“问题意识”,不要选不是问题的问题来研究。}
  \item 选题背景及意义:论述清楚为什么选择这个题目来研究,即阐述该研究对学科发展的贡献、对国计民生的理论与现实意义等。
  \item 文献综述:对本研究主题范围内的文献进行详尽的综合述评,“述”的同时一定要有“评”,指出现有研究状态,仍存在哪些尚待解决的问题,讲出自己的研究有哪些探索性内容。
  \item 研究方法:讲清论文所使用的学术研究方法。
  \item 论文结构安排:介绍本论文的写作结构安排。
\end{itemize}

