% !TeX root = ../thuthesis-example.tex

\begin{acknowledgements}
  首先衷心感谢我学术道路上的引路人某某副教授和某某教授。在本硕阶段,刘导的指导和建议教会了我站在学术最前沿思考问题,批评和教诲让我建立了正直、严谨的科研观,逐步提升自己的科研素养。我的第一篇较高引用的论文就是在刘导的手把手指导下发表的,从那以后才建立了学术自信。在博士阶段,某某老师的无微不至的指导让我同样如沐春风,教会了我如何将自己的工作打磨得更为系统和扎实。两位导师的言传身教将使我终生受益。我也感谢实验室提供的资源,让我有机会接触工业级别的预训练和微调,为我的职业生涯打下了坚实的基础。
  
  在实验室同门的帮助下,我产出了学术生涯中很多重要的工作。在硕博阶段,首先要感谢周界、丁宁、韩旭三位师兄,他们依次在我硕博生涯的不同阶段给予我直接的指导和带路。没有他们,我的科研之旅会坎坷很多。我还要感谢我的论文合作者们,包括汪华东、林衍凯、崔淦渠、张正彦、黄宇飞、秦禹嘉、肖朝军、曾哲妮、胡锦毅、张新荣、贺超群、陈英发等同学,和他们的讨论让我受益匪浅,没有他们就不会有一篇篇论文从想法变成实现。

  在清华计算机系以外,我要感谢在MILA访学期间遇到的师长们。在唐建老师的指导下,我完成了学术生涯的第一篇论文。在投稿期间,唐老师、瞿锰学长等人的耐心和帮助让我坚持了下去。此外还要感谢清华大学物理系以及其师长们,物理系的自信、主动、交流的文化,自由包容的学术氛围让我有机会接触计算机和人工智能领域的知识,并最终选定人工智能方向作为一生为之奋斗的方向。在物理系学习的扎实的数理锻炼也为后续的工作打下了坚实的基础。
  
  特别感谢汪佳茵同学的悉心照顾和大力支持,是她在我忙碌的时候提醒我锻炼,在我失落的时候给我鼓劲,在我取得一些进展的时候陪我庆祝,没有她我的博士生涯不会如此多彩。她的耐心细致和清晰的逻辑规划能力帮我避免了很多困难,弥补了我自己的不足。

  最后,我要向我的父母表达最深切的感谢和敬意。他们虽然不了解我的研究内容,却始终是我最坚定的支持者。在我迷茫时,是他们鼓励我追随内心选择自己热爱的道路;在我疲惫时,是他们的关心和慰问给了我继续前行的动力。他们给了我轻松包容的成长环境,让我虽然在大学之前只知道刷题,但仍保留了敢于想象、勇于创新的天性和天赋。没有这样的家庭环境,就不会有我今天的一切成就。

  此外在博士生阶段,我有幸承蒙国家自然科学基金资助,特此致谢。
\end{acknowledgements}
