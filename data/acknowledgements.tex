% !TeX root = ../thuthesis-example.tex

\begin{acknowledgements}
  首先衷心感谢我学术道路上的引路人刘知远副教授和孙茂松教授。在本硕阶段,刘导的指导和建议教会了我站在学术最前沿思考问题,批评和教诲让我建立了正直、严谨的科研观,逐步提升自己的科研素养。我的第一篇较高引用的文章的论文就是在刘导的手把手指导下发表的,从那以后才建立了学术自信。在博士阶段,孙茂松老师的无微不至的指导让我同样如沐春风,教会了我如何将自己的工作打磨得更为体系和扎实。两位导师的言传身教将使我终生受益。我也感谢实验室提供的资源,让我有机会接触工业级别的预训练和微调,为我的职业生涯打下了坚实的基础。
  
  在实验室同门的帮助下,我产出了学术生涯中很多重要的工作。在硕博阶段,首先要感谢周界、丁宁、韩旭三位师兄,他们依次在我硕博生涯的不同阶段给予我直接的指导和带路。没有他们,我的科研之旅会坎坷很多。我还要感谢我的论文合作者们,包括汪华东、林衍凯、崔淦渠、张正彦、黄宇飞、秦禹嘉、肖朝军、曾哲妮、胡锦毅、张新荣、贺超群、陈英发等同学,和他们的讨论让我受益匪浅,没有他们就不会有一篇篇论文从想法变成实现。

  在清华计算机系以外,我要感谢在MILA访学期间遇到的师长们。在唐建老师的指导下,完成了学术生涯的第一篇论文。在投稿期间,唐老师、瞿锰学长等人的耐心和帮助让我坚持了下去。此外还要感谢清华大学物理系以及其师长们,物理系的自信、主动、交流的文化,自由包容的学术氛围和交流氛围让我有机会接触计算机和人工智能领域的知识,并最终选定人工智能方向作为一生为之奋斗的方向。在物理系学习的扎实的数理锻炼也为后续的工作打下了坚实的基础。
  
  特别感谢汪佳茵同学的悉心照顾和大力支持,是她在我忙碌的时候提醒我锻炼,在我失落的时候给我鼓劲,在我取得一些进展的时候陪我庆祝,没有她我的博士生涯不会如此多彩。她的耐心细致和清晰的逻辑规划能力帮我避免了很多困难,弥补了我自己的不足。

  最后,我要向我的父母表达最深切的感谢和敬意。他们虽然不了解我的研究内容,却始终是我最坚定的支持者。在我迷茫时,是他们鼓励我追随内心选择自己热爱的道路;在我疲惫时,是他们的关心和慰问给了我继续前行的动力。他们给了我轻松包容的成长环境,让我虽然在大学之前只知道刷题,但仍保留了敢于想象,勇于创新的天性和天赋。没有这样的家庭环境,就不会有我今天的一切成就。

  此外在博士生阶段,我有幸承蒙国家自然科学基金资助,特此致谢。
\end{acknowledgements}
